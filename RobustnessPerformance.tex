\documentclass[12pt, a4paper, openany]{report}

\def\VersionRapport{1.0}

\usepackage[utf8]{inputenc} % un package
\usepackage[T1]{fontenc}      % un second package
\usepackage[francais]{babel}  % un troisième package
\usepackage{layout}
\usepackage[top=2.7cm, bottom=2.5cm, left=3.5cm, right=3cm]{geometry}
\usepackage{setspace}

\frenchbsetup{StandardLists=true} % à inclure si on utilise \usepackage[french]{babel}
%\usepackage{enumitem}
\usepackage[shortlabels]{enumitem}
\usepackage{amssymb}

\usepackage{color}
\usepackage{listings}
\definecolor{dkgreen}{rgb}{0,0.6,0}
\definecolor{gray}{rgb}{0.5,0.5,0.5}
\definecolor{mauve}{rgb}{0.58,0,0.82}

\lstset{frame=tb,
  language=Java,
  aboveskip=3mm,
  belowskip=3mm,
  showstringspaces=false,
  columns=flexible,
  basicstyle={\small\ttfamily},
  numbers=none,
  numberstyle=\tiny\color{gray},
  keywordstyle=\color{blue},
  commentstyle=\color{dkgreen},
  stringstyle=\color{mauve},
  breaklines=true,
  breakatwhitespace=true,
  tabsize=3
}

\usepackage{multirow} % pour les tableaux
\usepackage[table]{xcolor} % pour les tableaux

\usepackage{verbatim}
\usepackage{moreverb}
\usepackage{url}
\usepackage{pst-all}
\usepackage{eso-pic,graphicx}
\usepackage{caption} 
\usepackage[colorlinks=true,urlcolor=blue,linkcolor=red]{hyperref}
\usepackage{array}
\usepackage[toc,page]{appendix}
\usepackage[off]{auto-pst-pdf}
\usepackage{hyperref} % pour le sommaire table des matières
\AddThinSpaceBeforeFootnotes % à insérer si on utilise \usepackage[french]{babel}
\FrenchFootnotes % à insérer si on utilise \usepackage[french]{babel}
\usepackage{fancyhdr}
\pagestyle{headings}
\usepackage{pifont}  %pour les puces
\usepackage{amsmath} %pour les puces

\renewcommand{\appendixpagename}{Annexes}
\renewcommand{\appendixtocname}{Annexes}

\title{Theme: Rapport Performance & Robustesse}
\author{REBOUT \bsc{Mehenna}}
\author{BOUYOUCEF \bsc{Farid}}
\date{2018-2019}



%new
\newcommand{\HRule}{\rule{\linewidth}{0.5mm}}


\begin{document}

%\selectlanguage{francais}
\pagenumbering{arabic} 

\makeatletter
  \begin{titlepage}
  

  \begin{sffamily}
   \begin{center}

    % Upper part of the page. The '~' is needed because \\
    % only works if a paragraph has started.
    \includegraphics[scale=0.5]{Logo_UT3.jpg}~\\[1.5cm]

    \textsc{\LARGE Master 1 EEA ISTR/RODECO  }\\[2cm]

    \textsc{\Large Rapport Performance et Robustesse}\\[1.5cm]

    % Title
    \HRule \\[0.4cm] % saut de ligne
    { \huge \bfseries TP 1 Analyse et performances des systèmes linéaires : Asservissement d'un système à trois bacs d'eau\\[0.4cm] }

    \HRule \\[1cm]   % sous de ligne 
    \includegraphics[scale=0.1]{logomaster.jpg}
    \\[1cm]

    % Author and supervisor
    \begin{minipage}{0.4\textwidth}
      \begin{flushleft} \large
         \textsc{\emph {Réalisés par:} \\REBOUT Mehenna}\\
         \textsc{BOUYOUCEF Farid}   
          \newline
          Promotion 2018-2019 \\
      \end{flushleft}
    \end{minipage}
    \begin{minipage}{0.4\textwidth}
      \begin{flushright} \large
        \emph{Tuteur :}  \textsc{M DUROLA}\\
        \emph{Responsable de la Formation:} \textsc{M GOUAISBAUT}
      \end{flushright}
    \end{minipage}

    \vfill

    % Bottom of the page
    {\large Octobre 2018}

  \end{center}
  \end{sffamily}      
          
  \end{titlepage}
  
\makeatother



% *********************** Remerciements *****************
\chapter*{Remerciements}
 \addcontentsline{toc}{chapter}{Remerciements}

  Je tiens à exprimer ma profonde gratitude à mon promoteur, monsieur XXXX pour m'avoir encadré et guidé tout au long de l'année scolaire et mon mémoire, pour ses conseils judicieux et minutieusement prodigués.\\
  
  Aussi je tiens à lui reconnaître le temps précieux qu’il m'a consacré \\
  
   Que les membres du jury trouvent ici mes remerciements les plus vifs pour avoir accepté d’honorer par leur jugement mon travail.\\
   
   Mes sincères sentiments vont à tous ceux qui, de près ou de loin, ont contribué à la réalisation des mes études supérieurs, en particulier ma chères familles et mes amis (es).\\
   
   
%*********************** somaire **************
\renewcommand{\contentsname}{Sommaire}
\tableofcontents
%*********************** listes des figures **************
%\listoffigures
%*********************** listes des tableaux **************
%\listoftables



%*********************** INTRODUCTION **************
\chapter*{Introduction}
\addcontentsline{toc}{chapter}{Introduction}

Dans ce mini projet on va manipuler la commande robuste d’un système non linéaire autour d’un point de fonctionnement, ce que va nous permettre d’utiliser les loi de commande robuste comme le Loop-shaping.\\

 
 \paragraph{Définition de la Robustesse :}  capacité à rester opérationnel malgré de grandes amplitudes de conditions d’activité.
L’amélioration de la robustesse permet de retarder le moment où une dégradation sera réellement grave du point de vue de ceux qui utilisent le service (ne plus avoir de réseau téléphonique, par exemple).
Cela passe notamment par la mise en place de redondances pour permettre à la QoS de rester acceptable par les utilisateurs même au-delà des conditions techniques « normales » de fonctionnement. Il y aura toujours une limite, je vous l’accorde, mais les actions renforcement permettent de retarder le moment où vous déclencherez la cellule de crise opérationnelle \cite{ref1}
    
    %******************* PROBLEMATIQUE**********
 \chapter{Présentation du procédé}                                                

\begin{center}
    \includegraphics[scale=1]{troisBacs.png}
    \captionof{figure}{\textit{Procédé trois bacs \cite{ref2}}}
    \label{fig1}
    \end{center}

Le système représenté sur la figure 1 est composé de trois bacs cylindriques en plexiglas de section S. Ces trois bacs sont disposes en série (de gauche à droite, on trouve les bacs 1, 3 et 2) et sont relies par des tuyaux d’écoulement de section Sn. Le dernier bac (2) se vide par un cylindre également de section Sn dans un réservoir situé sous les bacs.
Deux pompes de débit Q1(t) et Q2(t) permettent de remplir respectivement les bacs 1 et 2 avec l’eau récupérée dans le  réservoir, le système fonctionnant en circuit fermé.
Lors de cette manipulation, nous allons nous intéresser à la régulation de niveau d’eau dans les bacs. Les niveaux, notés H1(t), H3(t) et H2(t) ne peuvent excéder la valeur de 0,6 m.
Les valeurs données par le constructeur sont : $S = 0,0154 \hspace{2mm} m2$ et $Sn = 5*10^{5} \hspace{2mm} m2$.
Une armoire de commande, reliée au procédé, permet de commander (en tension) les deux pompes et de mesurer (en tension) les trois hauteurs d’eau dans les bacs par l'intermédiaire de capteurs de niveau ; ces capteurs sont basés sur la mesure de la pression dans les bacs.

\paragraph{Cahier de charge :} 

 \begin{enumerate}[(a)]
   \item Le système sera stable en boucle fermée
   \item Une erreur de position nulle
   \item Une erreur de vitesse limitée à 1 i.e. lorsque l'entrée de consigne est une rampe de pente 1
   \item Une convergence vers la consigne en moins de 3 seconde, sans oscillations, ni dépassement.
   \item Un rejet de la perturbation de fuite.
   \item Un rejet du bruit de mesure au-delà de 100 Hz d’au moins de 60 dB par.
 \end{enumerate}
	
%*********************** Problématique **************
\chapter{ Analyse d'une commande proportionnelle intégrales}
 %\addcontentsline{toc}{chapter}{Analyse d'une commande proportionnelle intégrales}
 \chaptermark{Cmd proportionnelle intégrales}
 
 \section{Schéma bloc de l'asservissement}  %**** Question 1 ***** 
   % captures d’écrans 
   On donne la fonction de transfert entre le débit d'entrée  $q_{e}(p)$ et la sortie  $h_{1}(p)$ par: \\
   \begin{center}
   $G(p)=\frac {h_{1}(p)}{q_{e}(p)}=\frac{16p+20}{p^{3}+7p^{2}+12.75p+6.75}$   
   \end{center}
   
   On considère le premier correcteur de la forme : \\
   \begin{center}
   $U(p)=K\frac {1+\tau_{i}p}{\tau_{i}p}\xi(p)$ 
   \\[2cm]  
   \end{center}  
   
   \begin{center}
    \includegraphics[scale=0.8]{schemabloc.png}
    \captionof{figure}{\textit{Schéma bloc pour un asservissement d’une commande proportionnel intégrale}}
    \label{fig2}
   \end{center}
  
   
  
 \section{Débit de fuite est constant}  %**** Question 2  *****
 
 
 On utilisant les lois de la Mécanique Des Fluides, la pression est proportionnelle au débit volumique, est on constatant que l'eau est incompressible donc le volume ne varie pas par rapport au temps, d’où le débit volumique soit constant, l'hypothèse est valable. 
 
 
 \section{Le Diagramme asymptotique de $K(p)$ }  %**** Question 3  *****
 
 On a  $G(p)=\frac {U(p)}{\xi(p)}=K\frac {1+\tau_{i}p}{\tau_{i}p}$\\
 
 est on obtient le diagramme suivant:\\
 
 ilaqagh MATLAB pour le diagramme ........
 
 
 \section{Les spécifications} %***** Question 4  *******

 D’après la première commande on a un correcteur Proportionnel Intégrale, on obtient alors un déphasage de -$\pi$ en base fréquence et aussi un gain infini, et selon le cahier de charge pour avoir une convergence vers la consigne en moins de 3 seconde, il nous faut alors une bande passante élevée, ainsi un effet intégrateur afin d'avoir une erreur de position nulle. (correcteur PI $\Rightarrow$ erreur de position nulle)
 
 
 \section{Les contraintes sur le Gabarit 1 ainsi sur le Gabarit 2}  %**** Question 5  *****
 
 Dans ces deux Gabarits, on va déterminer ses contraintes sur la fonction de sensibilité permettant de satisfaire les spécifications b) et c) de notre cahier de charge.\\[0.1cm]\\
  %%***************** Gabarit 1 *********
 \begin{itemize}[label=\ding{108},font=\small\color{black}]
\item Pour une erreur de position nulle on a:\\
$\xi_{p}= \underset{p\longrightarrow 0}{lim}\hspace{2mm} p\hspace{2mm}\xi(p)$ \\
        =$\underset{p\longrightarrow 0}{lim}\hspace{2mm} p\hspace{2mm} S(p)\hspace{2mm} r(p)$ \hspace{2cm} tel que: \hspace{0.5cm}$r(p)=\frac {1}{p}$ \\
        =$\underset{p\longrightarrow 0}{lim}\hspace{2mm} p\hspace{2mm} S(p)\hspace{2mm} \frac {1}{p} $\\ 
        =S(0)=0\\
        \\[1mm]\\
        Si on choisit $w_{1}(p)=\frac {1}{p} $ \hspace{1cm} tel que :\hspace{2mm}|$w_{1}(p).S(p)|<1$\\
        $\Rightarrow$ |$S(p)|<p$ \hspace{3mm} et on remplaçant dans la limite on obtient : \\ 
        $\xi_{p}=\underset{p\longrightarrow 0}{lim}\hspace{2mm} p\hspace{2mm} p \hspace{2mm} \frac {1}{p} $\\
        %**\\[0.2cm]
        $=\underset{p\longrightarrow 0}{lim}\hspace{2mm} p\hspace{2mm}$\\
        =\hspace{2mm}0  \\
        \\[0.5cm]\\
       $\Rightarrow$  \hspace{2mm} $|S(jw)|_{H}$<$jw$  \hspace{2mm} tel que : $|S(jw)|_{H}$ est la valeur maximum que peut atteindre le module de $S(p)$ \\
       donc : $w_{1}(p)=\frac {1}{p} $ \hspace{2mm} est un gabarit nécessaire pour une erreur de position nulle.\\
  \end{itemize}
    %*********** Gabarit 2  *******
  \begin{itemize}[label=\ding{108},font=\small\color{black}]
  \item Pour une erreur de vitesse limitée à 1 lorsque l'entrée de consigne est une rampe de pente 1 on a :\\ 
 $\xi_{v}= \underset{p\longrightarrow 0}{lim}\hspace{2mm} p\hspace{2mm}\xi(p)$ \\
    \hspace{2cm}=$\underset{p\longrightarrow 0}{lim}\hspace{2mm} p\hspace{2mm} S(p)\hspace{2mm} r(p)$ \hspace{2cm} tel que: \hspace{0.5cm}$r(p)=\frac {1}{p^{2}}$ \\
     \\[1mm]\\
        Si on choisit $w_{2}(p)=\frac {1}{p} $ \hspace{1cm} tel que :\hspace{2mm}|$w_{2}(p).S(p)|<1$\\
        $\Rightarrow$ |$S(p)|<p$ \hspace{3mm} et on remplaçant dans la limite on obtient alors : \\ 
        $\xi_{v}=\underset{p\longrightarrow 0}{lim}\hspace{2mm} p\hspace{2mm} p \hspace{2mm} \frac {1}{p^{2}} $\\
        $=\underset{p\longrightarrow 0}{lim}\hspace{2mm} \frac {p^{2}}{p^{2}}\hspace{2mm}$\\
        =\hspace{2mm}1 $\leq$1 \\
        \\[5mm]\\
        $\Rightarrow$  \hspace{2mm} $|S(jw)|_{H}$<$jw$  \hspace{2mm} tel que : $|S(jw)|_{H}$ est la valeur maximum que peut atteindre le module de $S(p)$ \\
       donc : $w_{2}(p)=\frac {1}{p} $ \hspace{2mm} est un gabarit nécessaire pour une erreur de vitesse limitée à 1 lorsque l'entrée de consigne est une rampe de pente 1.\\
  \end{itemize} 

\section{La spécification sur la vitesse de convergence ainsi le Gabarit 3 pour la sensibilité}  %**** Question 6 et 7 *****

D'après notre cahier de charge, sur le système on veut avoir une convergence vers la consigne en mois de 3 seconde, et pour que le système soit plus rapide selon la bande passante on pose alors comme une contrainte:\\
 $|T(jw)|_{db}-|T(0)|_{db} \hspace{0.2cm} > \hspace{0.2cm} 3_{db}$\\ 
 \\[0.07cm]\\
 et pour avoir une condition suffisante, alors d'après la contrainte supposée ci-dessus on a :\\
 \\[1mm]\\
 20.log($|T(jw)|_{db}-|T(0)|_{db}$) \hspace{0.2cm} > \hspace{0.2cm} 20.logS($3_{db}$)\\
 
$\Rightarrow\hspace{0.2cm} \frac {|T(jw)|}{|T(0)|} \hspace{0.2cm} > \hspace{0.2cm} 10^{3/20} \hspace{0.2cm}=1/\sqrt{2} \hspace{0.5cm} avec : \hspace{0.2cm} |T(0)|=1$ \\

$\Rightarrow\hspace{0.2cm} |T(jw)| \hspace{0.2cm} > \hspace{0.2cm} 1/\sqrt{2}$ \\

on choisit alors :$\hspace{0.3cm} |S(jw)|\hspace{0.2cm}\leq \hspace{0.2cm} 1-(1/\sqrt{2}).......(1)$ \\
 
et on a aussi: $\hspace{0.3cm} |T(jw)+S(jw)|\hspace{0.2cm}= \hspace{0.2cm} 1$\\

alors: $\hspace{0.3cm} |T(jw)|+|S(jw)|\hspace{0.2cm}\geq \hspace{0.2cm} 1........(2)$\\  

de (1) on a : $1/\sqrt{2}\leq 1-|S(jw)|........(3)$ \\

et aussi de (2) $|T(jw)|\geq 1-|S(jw)|.......(4)$ \\

alors de (3) et de (4) on obtient: \\
 
 $\hspace{0.3cm} 1/\sqrt{2}\hspace{0.2cm} \leq 1-|S(jw)| \hspace{0.2cm}\leq |T(jw)|$ \\
 
 $\Rightarrow \hspace{0.3cm} |T(jw)| \hspace{0.2cm} \geq 1/\sqrt{2}$ \\
 
donc la condition est suffisante, et cela on va calculer le module en dB de la fonction de sensibilité selon ce choit : \\

on a alors : $|S(jw)|\hspace{0.2cm}\leq 1-(1/\sqrt{2}) \hspace{0.3cm} , \forall w\in[0,w_{c}] $ \\ 

 $20.log(1-(1/\sqrt{2})\hspace{0.2cm} \simeq \hspace{0.2cm} -10.67$  \\
 
 $\Rightarrow |S(jw)|_{db}\hspace{0.2cm} \leq \hspace{0.2cm} -10.67 \hspace{0.3cm} , \forall w\in[0,w_{c}] $  \\
 
 donc : le Gabarit 3 est nécessaire pour la fonction de sensibilité quelque soit $w\in[0,w_{c}]. $   
 
 
\section{La spécification selon les oscillations (Gabarit 4)}  %**** Question 8 ***** 

selon notre cahier de charge, il est indiqué que des oscillations ne sont pas souhaitables
 
   
 



   
%*********************** Loop-shaping **************
\chapter{Réalisation d'une commande Loop-shaping}
 


% *********************** Conclusion *****************
\chapter*{Conclusion}
\addcontentsline{toc}{chapter}{Conclusion}


%Bibliographie 
\bibliographystyle{alpha}
\bibliography{biblio}
\end{document}





