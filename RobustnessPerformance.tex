\documentclass[12pt, a4paper, openany]{report}

\def\VersionRapport{1.0}

\usepackage[utf8]{inputenc} % un package
\usepackage[T1]{fontenc}      % un second package
\usepackage[francais]{babel}  % un troisième package
\usepackage{layout}
\usepackage[top=2.7cm, bottom=2.5cm, left=3.5cm, right=3cm]{geometry}
\usepackage{setspace}

\frenchbsetup{StandardLists=true} % à inclure si on utilise \usepackage[french]{babel}
\usepackage{enumitem}
\usepackage{amssymb}

\usepackage{color}
\usepackage{listings}
\definecolor{dkgreen}{rgb}{0,0.6,0}
\definecolor{gray}{rgb}{0.5,0.5,0.5}
\definecolor{mauve}{rgb}{0.58,0,0.82}

\lstset{frame=tb,
  language=Java,
  aboveskip=3mm,
  belowskip=3mm,
  showstringspaces=false,
  columns=flexible,
  basicstyle={\small\ttfamily},
  numbers=none,
  numberstyle=\tiny\color{gray},
  keywordstyle=\color{blue},
  commentstyle=\color{dkgreen},
  stringstyle=\color{mauve},
  breaklines=true,
  breakatwhitespace=true,
  tabsize=3
}

\usepackage{multirow} % pour les tableaux
\usepackage[table]{xcolor} % pour les tableaux

\usepackage{verbatim}
\usepackage{moreverb}
\usepackage{url}
\usepackage{pst-all}
\usepackage{eso-pic,graphicx}
\usepackage{caption} 
\usepackage[colorlinks=true,urlcolor=blue,linkcolor=red]{hyperref}
\usepackage{array}
\usepackage[toc,page]{appendix}
\usepackage[off]{auto-pst-pdf}
\usepackage{hyperref} % pour le sommaire table des matières
\AddThinSpaceBeforeFootnotes % à insérer si on utilise \usepackage[french]{babel}
\FrenchFootnotes % à insérer si on utilise \usepackage[french]{babel}
\usepackage{fancyhdr}
\pagestyle{headings}

\renewcommand{\appendixpagename}{Annexes}
\renewcommand{\appendixtocname}{Annexes}

\title{Theme: Rapport Performance & Robustesse}
\author{REBOUT \bsc{Mehenna}}
\author{BOUYOUCEF \bsc{Farid}}
\date{2018-2019}



%new
\newcommand{\HRule}{\rule{\linewidth}{0.5mm}}


\begin{document}

\selectlanguage{francais}
\pagenumbering{arabic} 

\makeatletter
  \begin{titlepage}
  

  \begin{sffamily}
   \begin{center}

    % Upper part of the page. The '~' is needed because \\
    % only works if a paragraph has started.
    \includegraphics[scale=0.5]{Logo_UT3.jpg}~\\[1.5cm]

    \textsc{\LARGE Master 1 EEA ISTR/RODECO  }\\[2cm]

    \textsc{\Large Rapport Performance et Robustesse}\\[1.5cm]

    % Title
    \HRule \\[0.4cm] % sous de ligne
    { \huge \bfseries TP 1 Analyse et performances des systèmes linéaires : Asservissement d'un système à trois bacs d'eau\\[0.4cm] }

    \HRule \\[1cm]   % sous de ligne 
    \includegraphics[scale=0.1]{logomaster.jpg}
    \\[1cm]

    % Author and supervisor
    \begin{minipage}{0.4\textwidth}
      \begin{flushleft} \large
         \textsc{\emph {Réalisés par:} \\REBOUT Mehenna}\\
         \textsc{BOUYOUCEF Farid}   
          \newline
          Promotion 2018-2019 \\
      \end{flushleft}
    \end{minipage}
    \begin{minipage}{0.4\textwidth}
      \begin{flushright} \large
        \emph{Tuteur :}  \textsc{M DUROLA}\\
        \emph{Responsable de la Formation:} \textsc{M GOUAISBAUT}
      \end{flushright}
    \end{minipage}

    \vfill

    % Bottom of the page
    {\large Octobre 2018}

  \end{center}
  \end{sffamily}      
          
  \end{titlepage}
  
\makeatother



% *********************** Remerciements *****************
\chapter*{Remerciements}
 \addcontentsline{toc}{chapter}{Remerciements}

  Je tiens à exprimer ma profonde gratitude à mon promoteur, monsieur XXXX pour m'avoir encadré et guidé tout au long de l'année scolaire et mon mémoire, pour ses conseils judicieux et minutieusement prodigués.\\
  
  Aussi je tiens à lui reconnaître le temps précieux qu’il m'a consacré \\
  
   Que les membres du jury trouvent ici mes remerciements les plus vifs pour avoir accepté d’honorer par leur jugement mon travail.\\
   
   Mes sincères sentiments vont à tous ceux qui, de près ou de loin, ont contribué à la réalisation des mes études supérieurs, en particulier ma chères familles et mes amis (es).\\
   
   
%*********************** somaire **************
\renewcommand{\contentsname}{Sommaire}
\tableofcontents
%*********************** listes des figures **************
%\listoffigures
%*********************** listes des tableaux **************
%\listoftables



%*********************** INTRODUCTION **************
\chapter*{Introduction}
\addcontentsline{toc}{chapter}{Introduction}
 
  Une des raisons de construire des logiciels est pour une utilité utilisateurs.bla bla
                                                      



%*********************** Problématique **************
\chapter{ Analyse d'une commande proportionnelle intégrales}
 %\addcontentsline{toc}{chapter}{Analyse d'une commande proportionnelle intégrales}
\section{Schéma bloc de l'asservissement} 
 	jsk vive jsk  ajl,celkj,cx   
   Depuis l’apparition de la nécessité de. \\
   
   Les systèmes .... bla bla... ajoutant à cela quelques problèmes connus:
   
    \begin{itemize}[label=$\square$]
      \item  Les applications.
      \item  Les problèmes de. 
      \item  L'ajout de.
      \item  Les applications.
  \end{itemize}
  
   Malheureusement beaucoup d'entreprises ont bla bla. \\
   
   De nos jours....
   
   
%*********************** Performance et Robustesse **************
\chapter{Réalisation d'une commande Loop-shaping}
 La Performance c’est tout d’abord un blaba.\\
 
 La Robustesse c’est tout d’abord un blaba.\\ \\
 % captures d’écrans 
 \begin{center}
   \includegraphics[scale=0.7]{image2.png}
   \captionof{figure}{\textit{Optimisation Robustesse}}
   \label{fig1}
 \end{center}
 
 La figure au-dessus a ... bla bla ....\\
 
 Les bla bla ....\\
 
 Néanmoins, cette structure n’est jamais stable au fil du temps, ... bla bla.\\
  
 En court, il y a une différence entre ... bla bla ..., est fort envisageable qu’elle subira des changements dans le temps.
 
 \paragraph{Définition:}
  Bla bla ....

 \section{Section une}
  ggggggggggggggg



% *********************** Conclusion *****************
\chapter*{Conclusion}
\addcontentsline{toc}{chapter}{Conclusion}

\end{document}
